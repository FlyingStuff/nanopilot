\documentclass[a4paper]{paper}

\usepackage[utf8]{inputenc}
\usepackage{amsmath}
\usepackage[left=2.5cm,top=3cm,right=2.5cm,bottom=3cm,bindingoffset=0.cm]{geometry}
\usepackage{bm} % used for bold math symbols
\usepackage{parskip} % vertical space instead of indented paragraphs
\usepackage{hyperref}
\usepackage{graphicx}

\usepackage{biblatex}
\usepackage{filecontents}


\begin{filecontents}{references.bib}
@article{MarkleyAttitudeErrorRepr,
    author = {Markley, F. Landis},
    publisher = {Journal of Guidance, Control, and Dynamics, Vol. 26, No. 2 (2003), pp. 311-317.},
    year = {2003},
    title = {Attitude Error Representations for Kalman Filtering},
    url = {http://ntrs.nasa.gov/archive/nasa/casi.ntrs.nasa.gov/20020060647.pdf}
},
\end{filecontents}

\addbibresource{references.bib}


\author{
Patrick Spieler \\
\texttt{stapelzeiger@gmail.com}
}

\title{Attitude EKF Documentation}


\begin{document}
\maketitle
\hrule
\bigskip

% \begin{abstract}

% This documents the model used for the Extended Kalman Filter estimating attitude.

% \end{abstract}

% \tableofcontents

% \pagebreak


\section{Attitude Representation}

The attitude is represented as an attitude error with respect to a reference attitude. This is called Multiplicative EKF (MEKF). See \cite{MarkleyAttitudeErrorRepr}, (The quaternion multiplication and rotation matrix from quaternion are inverted with respect to the standard convention used here).


\begin{equation}
    q_b^I = q_{ref} \circ \delta q(\bm{a})
    \label{eq:delta_q}
\end{equation}


The attitude error $\delta q(\bm{a})$ is tracked by the EKF. It is defined such that it's expected value is a zero rotation. This allows the choice of a parametrization with only three variables without issues with singularity.

The chosen attitude error parametrization $\bm{a}$ follows \cite{MarkleyAttitudeErrorRepr}. It is defined to be twice the Gibbs vector (choosing a factor of two makes the covariance of $\bm{a}$ an angle covariance):

\begin{equation}
    \bm{a} = 2\bm{g} = 2 \frac{\delta q_v}{\delta q_0}
\end{equation}

\begin{equation}
    \delta q(\bm{a}) =
    \left\{
      \begin{array}{rcl}
        \delta q_{0} & = & \frac{2}{\sqrt{4+\lVert\bm{a}\rVert^2}} \\
        \delta q_{v} & = & \frac{\bm{a}}{\sqrt{4+\lVert\bm{a}\rVert^2}} \\
      \end{array}
    \right.
\label{eq:delta_q_from_a}
\end{equation}

Taking the time derivative of (\ref{eq:delta_q}):
\begin{equation}
\begin{split}
    \frac{d}{dt} q = \dot q
    &= \frac{d}{dt} (q_{ref} \circ \delta q)\\
    \dot q &= \dot{q_{ref}} \circ \delta q + q_{ref} \circ \dot{\delta q} \\
    \dot{\delta q} &= q_{ref}^{-1} \circ \dot q - q_{ref}^{-1} \circ \dot{q_{ref}} \circ \delta q \\  % (using quat-time-der)
    \dot{\delta q} &= \frac{1}{2} q_{ref}^{-1} \circ  q_{ref} \circ \delta q \circ \omega
        - \frac{1}{2} q_{ref}^{-1} \circ q_{ref} \circ \omega_{ref} \circ \delta q \\
    \dot{\delta q} &= \frac{1}{2} \delta q \circ \omega - \frac{1}{2} \omega_{ref} \circ \delta q \\
\end{split}
\label{eq:delta_q_dot}
\end{equation}

Taking the time derivative of $\bm{a}$
\begin{equation}
    \frac{d}{dt} \bm{a}(\delta q) = \frac{\partial \bm{a}}{\partial \delta q} \dot{\delta q}
    = \frac{2}{\delta q_0} \dot{\delta q_v} - \frac{2 \dot{\delta q_0}}{{\delta q_0}^2} \delta q_v
\end{equation}

Substituting $\delta q$ and $\dot{\delta q}$ using (\ref{eq:delta_q_from_a}) and (\ref{eq:delta_q_dot}):

\begin{equation}
    \begin{split}
    \dot{\bm{a}} &=
        \sqrt{4+\lVert\bm{a}\rVert^2}
        \left\{ \frac{1}{2}\delta q \circ \bar \omega - \frac{1}{2} \bar \omega_{ref}\circ \delta q \right\}_v
        -  \frac{1}{2} \sqrt{4+\lVert\bm{a}\rVert^2} \left\{ \frac{1}{2}\delta q \circ \bar \omega - \frac{1}{2} \bar \omega_{ref}\circ \delta q \right\}_0 \bm{a} \\
    &= \left\{
            \frac{1}{2}
            \left[ \begin{array}{c} 2 \\ \bm{a} \end{array} \right]
            \circ \bar \omega
            - \frac{1}{2}
            \bar \omega_{ref} \circ
            \left[ \begin{array}{c} 2 \\ \bm{a} \end{array} \right]
        \right\}_v
        - \frac{1}{2} \left\{
            \frac{1}{2}
            \left[ \begin{array}{c} 2 \\ \bm{a} \end{array} \right]
            \circ \bar \omega
            - \frac{1}{2}
            \bar \omega_{ref} \circ
            \left[ \begin{array}{c} 2 \\ \bm{a} \end{array} \right]
        \right\}_0 \bm{a} \\
\end{split}
\end{equation}

Expanding the quaternion multiplication using the vector operations we have:

\begin{equation}
    \begin{split}
    \dot{\bm{a}} &=
        \frac{1}{2} (2 \bm{\omega} + \bm{a} \times \bm{\omega})
        - \frac{1}{2} (2 \bm{\omega}_{ref} + \bm{\omega}_{ref} \times \bm{a})
        - \frac{1}{2} \left(
            \frac{1}{2} \bm{a} \cdot \bm{\omega}
            - \frac{1}{2} \bm{\omega}_{ref} \cdot \bm{a}
        \right) \bm{a} \\
    &= \bm{\omega} - \bm{\omega}_{ref}
        - \frac{1}{2} (\bm{\omega} - \bm{\omega}_{ref}) \times \bm{a}
        - \bm{\omega}_{ref} \times \bm{a}
        - \frac{1}{4} ((\bm{\omega} - \bm{\omega}_{ref}) \cdot \bm{a}) \bm{a}
\end{split}
\label{eq:adot}
\end{equation}




\section{Gyroscope}

The output of the gyroscopes is modeled as the real body angular rate with additive time varying bias and white noise:

\begin{equation}
    \bm{\omega}_g(t) = \bm{\omega}(t) + \bm{b}(t) + \bm{n}_g(t)
\end{equation}

The bias is modeled as a random walk.
\begin{equation}
    \dot{\bm{b}}(t) = \bm{n}_b(t)
\end{equation}

\section{Direction Vector Measurement}

\subsection{Gravitational Acceleration}

\subsection{Magnetic Field}

\section{Quaternion Measurement}

Attitude measurement $\tilde{q}_b^I$:
\begin{equation}
    \delta q(\bm{a}=\bm{z}) = q_{ref}^{-1} \circ \tilde{q}_b^I
    \label{eq:q_measurement_def}
\end{equation}
Then:
\begin{equation}
    \bm{z} = h(\bm{a}) = \bm{a}
    \label{eq:q_measurement_fn}
\end{equation}

\begin{equation}
    H_a = I_{3\times3}
    \label{eq:q_measurement_H}
\end{equation}

The measurement noise $R_a$ is the covariance of the error angle.

\section{EKF covariance propagation}

For a filter tracking attitude error and gyro bias, the state $\bm{x}$ is:

\begin{equation}
    \bm{x} = \left[
        \begin{matrix}
            \bm{a}\\[0.3em]
            \bm{b}\\
        \end{matrix}
        \right]
\end{equation}

The expectation of the time derivative is:

\begin{equation}
    E (\bm{\dot x}) = \left[
        \begin{matrix}
            0_{3\times1}\\[0.3em]
            0_{3\times1}\\
        \end{matrix}
        \right]
\end{equation}

The update Jacobians are:

\begin{equation}
    F = E \left\{
        \left[
        \begin{matrix}
            \frac{\partial \dot{\bm{a}}}{\partial \bm{a}} & \frac{\partial \dot{\bm{a}}}{\partial \bm{b}}\\[0.3em]
            \frac{\partial \dot{\bm{b}}}{\partial \bm{a}} & \frac{\partial \dot{\bm{b}}}{\partial \bm{b}}\\
        \end{matrix}
        \right]
    \right\}
    = \left[
        \begin{matrix}
            -{[\bm{\omega}_{ref}]}_\times & -I_{3\times3}\\[0.3em]
            0_{3\times3} & 0_{3\times3}\\
        \end{matrix}
        \right]
\end{equation}

\begin{equation}
    G = E \left\{
        \left[
        \begin{matrix}
            \frac{\partial \dot{\bm{a}}}{\partial \bm{n}_g} & \frac{\partial \dot{\bm{a}}}{\partial \bm{n}_b}\\[0.3em]
            \frac{\partial \dot{\bm{b}}}{\partial \bm{n}_g} & \frac{\partial \dot{\bm{b}}}{\partial \bm{n}_b}\\
        \end{matrix}
        \right]
    \right\}
    = \left[
        \begin{matrix}
            -I_{3\times3} & 0_{3\times3}\\[0.3em]
            0_{3\times3} & I_{3\times3}\\
        \end{matrix}
        \right]
\end{equation}




\clearpage
\nocite{*}
\printbibliography[prefixnumbers={RD}]
\clearpage

\end{document}
